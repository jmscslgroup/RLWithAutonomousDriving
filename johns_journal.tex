\documentclass{article}
\begin{document}

\title{Research Paper Journal}
\author{John Nguyen}
\date{Last Update: 6/10}
\maketitle

\subsection*{Robust Reinforcement Learning for Autonomous Driving} 

\textbf{Novel Contribution:} \underline{Actor and multi-step critic model on a complex task (autonomous driving)}


\begin{itemize}
	\item Deep RL algorithm embedding actor critic architecture with multi-step returns
	
	\begin{itemize}
		\item actor critic architecture:
		\begin{enumerate}
			\item actor: the policy; agent
			\item critic: the value function
			\item Purpose is to reduce variance of policy gradient (formalizes trial and error choices)
		\end{enumerate}
		
		\item multi-step returns
		
		\begin{enumerate}
			\item Tempoeral difference update to control adn guide exploration
			\item ? (in normal words, what is "multi-step returns"?
		\end{enumerate}
		
	\end{itemize}	
	
	\item Improved actor and critic model using multi-step returns to obtain (marginally? [reviewer's notes]) better results than standard RL/
	
	\item *Model faced certain vulnerabilities when faced with unseen (not observable? not seen before inputs?)
	
	\begin{itemize}
		\item Human behavior, diversity of driving styles, complexity of scene perception
	\end{itemize}
	
	\item TODO List
	
	\begin{itemize}
		\item Test out other RL methods, such as deep Q-learning, Trust Region Policy Optimization, on other complex tasks.
		\item Impact of non-stationary environments on RL with multi-task learning problems. (Environment changes with respect to some probability distribution)
		\item Apply Adaptive Dynamic Programming (RL + Dynamic Programming); apporximate the dynamic programming solutions by estimating the cost function. 
		\item *Conext aware and meta learning strategies; generalizable and fast adapting algorithms (see Santaro et al. 2016; Ravi and Larochelle, 2017)
	\end{itemize}
	
	\item Time Differential (TD) Learning
	\begin{itemize}
		\item ? (what is this?)
	\end{itemize}
	
	\item Actor
	\begin{itemize}
		\item Gradient generated to optimize policy
		\item (+) Convergence (of learning) is guarenteed
		\item (-) High variance $\rightarrow$ slow learning 
	\end{itemize}
	
	\item Critic
	\begin{itemize}
		\item Appeoximates value function using TD variance
		\item (+) Lower variance $\rightarrow$ faster learning
		\item (-) Lack of reliable convergence and optimal solution may not exist.
	\end{itemize}
	
	\item Actor-Critic: Combines actor and critic models to iteraively and simultaneously update the model.
	
	\item Other Actor-Critic algorithms mentioned:
	\begin{itemize}
		\item FACRLN: 1 neural network that approximates critic and actor
		\item CACM: natural gradient for policy-critic algorithm
		\item DPG: Learned value estimate to train deterministic policy.
		\item A3C: Agents in parallel, data decorrelation and learning diversity
	\end{itemize}
	
	\item The above algorithms have only been tested on simple tasks.
	
	\item Authors implement the A2C algorithm
	
	\item A2C usess the Markov Decision Model
	
	\item Authors only tested urban driving
	
	\item Multi-step returns critic guides with value function
	
	\item Authors only tested in 2 different environments, yet they claim "superior generalization" of n-step A2C over regular RL.
	
	
	
\end{itemize}

\textbf{Review:} Value function, policy, gradient

\subsection*{NEW PAPER} 

\end{document}